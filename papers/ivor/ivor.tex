%\documentclass{article}
\documentclass[orivec,dvips,10pt]{llncs}

\usepackage{epsfig}
\usepackage{path}
\usepackage{url}
\usepackage{amsmath,amssymb} 

\newenvironment{template}{\sffamily}

\setlength{\parindent}{0pt}
\setlength{\parskip}{1ex}

\usepackage{graphics,epsfig}

\input{macros.ltx}
\input{library.ltx}

\NatPackage

\newcommand{\Ivor}{\textsc{Ivor}}
\newcommand{\Funl}{\textsc{Funl}}

\newcommand{\hdecl}[1]{\texttt{#1}}

\begin{document}

\title{\Ivor{}, a Proof Engine}
\author{Edwin Brady}

\institute{School of Computer Science, \\
 University of St Andrews, St Andrews, Scotland. \\ \texttt{Email: eb@dcs.st-and.ac.uk}.\\
\texttt{Tel: +44-1334-463253}, \texttt{Fax: +44-1334-463278} \vspace{0.1in}
}
 
\maketitle

\begin{abstract}
Dependent type theory has several practical applications in the fields
of theorem proving, program verification and programming language
design. \Ivor{} is a Haskell library designed to allow easy embedding
and extending of a type theory based theorem prover in a Haskell
application. In this paper, I give an overview of the library and show
how it can be used to implement formal systems such as a first order
logic theorem prover.  Further, I show how dependent type theory
can be used as a core representation for a functional programming
language, allowing an evaluator and theorem prover to be implemented
within the same framework.

\end{abstract}

\section{Introduction}

%\Ivor{} is a tactic-based theorem proving engine with a Haskell
%API. Unlike other systems such as \Coq{}~\cite{coq-manual} and
%Agda~\cite{agda}, the tactic engine is primarily intended to be used
%by programs, rather than a human operator. 

Type theory based theorem provers such as \Coq{}~\cite{coq-manual} and
\Agda{}~\cite{agda} have been used as tools for verification of programs
(e.g.~\cite{leroy-compiler,why-tool,mckinna-expr}), extraction of
correct programs from proofs (e.g.~\cite{extraction-coq})
and formal proofs of mathematical properties
(e.g.~\cite{fta,four-colour}).  However, these tools are designed with a
human operator in mind; the interface is textual which makes it
difficult for an external program to interact with them. 
In contrast, the \Ivor{} library is designed to provide an
implementation of dependent type theory (i.e. dependently typed
$\lambda$-calculus) and tactics for proof and
program development to a Haskell application programmer, via a stable,
well-documented and lightweight (as far as possible) API. The goal is
to allow: i) easy embedding of theorem proving tools in a Haskell
application; and ii) easy extension of the theorem prover with
\remph{domain specific} tactics, via a domain specific embedded
language (DSEL) for tactic construction.

%% \Coq{}
%% provides an extraction mechanism~\cite{extraction-coq} which generates
%% ML or Haskell code from a proof term, but this does not allow the easy
%% \remph{construction} of proof terms by an external tool. It is also
%% extensible to some extent, for example using a domain specific
%% language for creating user tactics, but the result is difficult to
%% embed in an external program.

%More
%recently, dependent types have been incorporated into programming
%languages such as Cayenne~\cite{cayenne-icfp}, DML~\cite{xi-thesis} and
%\Epigram{}~\cite{view-left,epigram-afp}.


%% have been used for several
%% large practical applications, including correctness proofs for a
%% compiler~\cite{leroy-compiler} and a computer assisted proof of the
%% four colour theorem~\cite{four-colour}. 

\subsection{Motivating Examples}

Many situations can benefit from a dependently typed proof and
programming framework accessible as a library from a Haskell program.
For each of these, by using an implementation of a well understood
type theory, we can be confident that the underlying framework is
sound. 

%% --- provided of course that the implementation itself is correct. 
%% Implementing
%% eliminates the need to prove that a language and proof system are consistent with
%% each other or that a special purpose proof framework is sound.

\begin{description}
\item[Programming Languages] 
Dependent type theory is a possible internal representation for a
functional programming language. 
%The core language of the Glasgow
%Haskell Compiler is \SystemF{}~\cite{core} --- dependent type theory
%generalises this by allowing types to be parametrised over values.
Correctness
properties of programs in purely functional languages can be proven by
equational reasoning, 
e.g. with Sparkle~\cite{sparkle} for the Clean language~\cite{clean}, or
Cover~\cite{cover} for translating Haskell into
\Agda{}~\cite{agda}. However these tools 
separate the language implementation from the theorem prover --- every
language feature must be translated into the theorem prover's
representation, and any time the language implementation is changed,
this translation must also be changed.
In section
\ref{example2}, we will see how \Ivor{} can be used to implement a
language with a built-in theorem prover, with a common representation
for both.

\item[Verified DSL Implementation] 

We have previously implementated a verified domain specific
language~\cite{dtpmsp-gpce} with \Ivor{}.  The abstract syntax tree of
a program is a dependent data structure, and the type system
guarantees that invariant properties of the program are maintained
during evaluation.  Using staging annotations~\cite{multi-taha}, such
an interpreter can be specialised to a translator. We are continuing
to explore these techniques in the context of resource aware
programming~\cite{dt-framework}.

\item[Formal Systems] 

A formal system can be modelled in dependent type theory, and
derivations within the system can be constructed and checked. 
A simple example is propositional logic --- the connectives
$\land$, $\lor$ and $\to$ are represented as types, and a theorem
prover is used to prove logical formulae.  Having an implementation of
type theory and an interactive theorem prover accessible as an API
makes it easy to write tools for working in a formal system, whether
for educational or practical purposes.  In section \ref{example1}, I
will give details of an implementation of propositional logic.

\end{description}

In general, the library can be used wherever formally certified code
is needed --- evaluation of dependently typed \Ivor{} programs is
possible from Haskell programs and the results can be inspected
easily.  Domain specific tactics are often required; e.g. an
implementation of a programming language with subtyping may require a
tactic for inserting coercions, or a computer arithmetic system may
require an implementation of Pugh's Omega decision
procedure~\cite{pugh-omega}.  \Ivor{}'s API is designed to make
implementation of such tactics as easy as possible.

In \Ivor{}'s dependent type system, types may be predicated on
arbitrary values. Programs and properties can be
expressed within the same self-contained system --- properties are
proved by construction, at the same time as the program is
written. The tactic language can thus be used not only for
constructing proofs but also for interactive program development.

%\subsection{Why Do We Need Another Theorem Prover?}

%Relationship to e.g. \Coq{}.


\section{The Type Theory, $\source$}

The core type theory of \Ivor{} is a strongly normalising dependent
type theory with inductive families~\cite{dybjer94}, similar to Luo's
UTT~\cite{luo94} or the Calculus of Inductive Constructions in
\Coq{}~\cite{coq-manual}. This language, which I call $\source$, is an
enriched lambda calculus, with the usual properties of subject
reduction, Church Rosser, and uniqueness of types up to
conversion. The strong normalisation property (i.e. that evaluation
always terminates) is guaranteed by allowing only primitive recursion
over strictly positive inductive datatypes. The syntax of this
language is as follows:

\DM{
\begin{array}{rllrll}
\vt ::= & \Type_i & (\mbox{type universes}) &
 \mid  & \vx & (\mbox{variable})\\
 \mid   & \fbind{\vx}{\vt}{\vt} & (\mbox{function space}) &
 \mid   & \lam{\vx}{\vt}.\:\vt & (\mbox{abstraction}) \\
 \mid   & \vt\:\vt & (\mbox{application}) &
 \mid & \LET\:\vx\defq\vt\Hab\vt\:\IN\:\vt & (\mbox{let binding}) \\
\end{array}
}

We may also abbreviate the function space $\fbind{\vx}{\vS}{\vT}$ by
$\vS\to\vT$ if $\vx$ is not free in $\vT$. The typing rules are given
in Figure \ref{typerules}.

\FFIG{
\begin{array}{c}
\Rule{\Gamma\proves\RW{valid}}
{\Gamma\vdash\Type_n\Hab\Type_{n+1}}\hspace*{0.1in}\mathsf{Type}
\hg
\Rule{(\vx\Hab\vS)\in\Gamma}
{\Gamma\vdash\vx\Hab\vS}\hspace*{0.1in}\mathsf{Var}
\hg
\Rule{(\vx\Hab\vS\defq\vs)\in\Gamma}
{\Gamma\vdash\vx\Hab\vS}\hspace*{0.1in}\mathsf{Val}
\\
\Rule{\Gamma\vdash\vf\Hab\fbind{\vx}{\vS}{\vT}\hg\Gamma\vdash\vs\Hab\vS}
{\Gamma\vdash\vf\:\vs\Hab\vT[\vs/\vx]} % \LET\:\vx\Hab\vS\:\defq\:\vs\:\IN\:\vT}
\hspace*{0.1in}\mathsf{App}
\\

\Rule{\Gamma;\vx\Hab\vS\vdash\ve\Hab\vT\hg\Gamma\proves\fbind{\vx}{\vS}{\vT}\Hab\Type_n}
{\Gamma\vdash\lam{\vx}{\vS}.\ve\Hab\fbind{\vx}{\vS}{\vT}}\hspace*{0.1in}\mathsf{Lam}
\\
\Rule{\Gamma;\vx\Hab\vS\vdash\vT\Hab\Type_n\hg\Gamma\vdash\vS\Hab\Type_n}
{\Gamma\vdash\fbind{\vx}{\vS}{\vT}\Hab\Type_n}\hspace*{0.1in}\mathsf{Forall}
\\

\Rule{\begin{array}{c}\Gamma\proves\ve_1\Hab\vS\hg
      \Gamma;\vx\defq\ve_1\Hab\vS\proves\ve_2\Hab\vT\\
      \Gamma\proves\vS\Hab\Type_n\hg
      \Gamma;\vx\defq\ve_1\Hab\vS\proves\vT\Hab\Type_n\end{array}
      }
{\Gamma\vdash\LET\:\vx\Hab\vS\defq\ve_1\:\IN\:\ve_2\Hab
   \vT[\ve_1/\vx]}   
%\Let\:\vx\Hab\vS\defq\ve_1\:\IN\:\vT}
\hspace*{0.1in}\mathsf{Let}
\\

\Rule{\Gamma\proves\vx\Hab\vA\hg\Gamma\proves\vA'\Hab\Type_n\hg
      \Gamma\proves\vA\converts\vA'}
     {\Gamma\proves\vx\Hab\vA'}
\hspace*{0.1in}\mathsf{Conv}
\end{array}
}
{Typing rules for $\source$}
{typerules}




%\section{Basic Usage}

\subsection{Concrete Syntax}

\subsection{The Shell}

\subsection{The API}

\section{The \Ivor{} Library}

%Given the basic operations defined in section \ref{holeops}, we can
%create a library of tactics. 

The \Ivor{} library allows the incremental, type directed development
of $\source$ terms.  In this section, I will introduce the basic
tactics available to the library user, along with the Haskell
interface for constructing and manipulating $\source$ terms. This
section includes only the most basic operations; the API is however
fully documented on the
web\footnote{\url{http://www.cs.st-andrews.ac.uk/~eb/Ivor/doc/}}.

\subsection{Definitions and Context}

The central data type is \hdecl{Context} (representing $\Gamma$ in the
typing rules), which is an abstract type holding information about
inductive types and function definitions as well as the current proof
state. All operations are defined with respect to the context. An
empty context is contructed with \hdecl{emptyContext :: Context}.

Terms may be represented several ways; either as concrete syntax (a
\texttt{String}), an abstract internal representation (\texttt{Term})
or as a Haskell data structure (\texttt{ViewTerm}). A typeclass
\hdecl{IsTerm} is defined, which allows each of these to be converted
into the internal representation. This typeclass has one method:

\begin{verbatim}
class IsTerm a where
    check :: Monad m => Context -> a -> m Term
\end{verbatim}

The \texttt{check} method parses and typechecks the given term, as
appropriate, and if successful returns the internal representation. 
Constructing a term in this way may fail (e.g. due to a syntax or
type error) so \texttt{check} is generalised over a monad
\hdecl{m} --- it may help to read \hdecl{m} as \hdecl{Maybe}.

In this paper, for the sake of readability we will use the syntax
described in section \ref{corett}, and assume an instance of
\hdecl{IsTerm} for this syntax.

Similarly, there is a typeclass for inductive families,
which may be represented either as concrete syntax or a Haskell data
structure.

\begin{verbatim}
class IsData a where
    addData :: Monad m => Context -> a -> m Context
\end{verbatim}

The \hdecl{addData} method adds the constructors and elimination
rules for the data type to the context. Again, we assume an instance
for the syntax presented in section \ref{indfamilies}.

The simplest way to add new function definitions to the context is
with the \hdecl{addDef} function. Definitions may not be recursive,
other than via the automatically generated elimination rules:

\begin{verbatim}
addDef :: (IsTerm a, Monad m) => Context -> Name -> a -> m Context
\end{verbatim}

However, \Ivor{} is primarily a library for constructing proofs; the
Curry-Howard correspondence identifies programs and proofs, and
therefore such definitions can be viewed as proofs; to prove a
theorem is to add a well-typed definition to the context.  We would
therefore like to be able to construct more complex proofs (and indeed
programs) interactively --- at the heart of \Ivor{} is a theorem
proving engine.

\subsection{Theorems}

In the \hdecl{emptyContext}, there is no proof in progress, so no
proof state --- the \hdecl{theorem} function creates a proof state in
a context. This will fail if there is already a proof in progress, or
the goal is not well typed.

\begin{verbatim}
theorem :: (IsTerm a, Monad m) => Context -> Name -> a -> m Context
\end{verbatim}

A proof state can be thought of as an incomplete term, i.e. a
term in the development calculus. For example, calling \hdecl{theorem}
with the name $\FN{plus}$ and type $\Nat\to\Nat\to\Nat$, an initial
proof state would be:

\DM{
\FN{plus}\:=\:\hole{\VV{plus}}{\Nat\to\Nat\to\Nat}
}

Proving a theorem proceeds by applying tactics to each unsolved hole
in the proof state. The system keeps track of which subgoals are still
to be solved, and a default subgoal, which is the next subgoal to be
solved. I will write proof states in the following form:

\DM{
\Rule{
\AR{
\mbox{\textit{bindings in the context of the subgoal $\vx_0$}} \\
\ldots\\
}
}
{
\AR{
\hole{\vx_0}{\mbox{\textit{default subgoal type}}} \\
\ldots \\
\hole{\vx_i}{\mbox{\textit{other subgoal types}}} \\
\ldots
}
}
}

Functions are available for querying the bindings in the context of
any subgoal. A tactic typically works on the bindings in scope and the
type of the subgoal it is solving.

When there are no remaining subgoals, a proof can be lifted into the
context with the \texttt{qed} function:

\begin{verbatim}
qed :: Monad m => Context -> m Context
\end{verbatim}

This function typechecks the entire proof. In practice, this check
should never fail --- the development calculus itself ensures that
partial constructions as well as complete terms are well-typed, so it
is impossible to build ill-typed partial constructions. However, doing
a final typecheck of a complete term means that the soundness of the
system relies only on the soundness of the typechecker for the core
language, e.g.~\cite{coq-in-coq}.  We are then free to implement tactics
in any way we like, safe in the knowledge that any ill-typed or
unsound constructions will be caught by the typechecker.

%% but if the tactics are correctly implemented this check will always succeed.

\subsection{Basic Tactics}

A tactic is an operation on a goal in the current system state; we
define a type synonym \hdecl{Tactic} for functions which operate as
tactics. Tactics modify system state and may fail, hence a tactic
function returns a monad:

\begin{verbatim}
type Tactic = forall m . Monad m => Goal -> Context -> m Context
\end{verbatim}

A tactic operates on a hole binding, specified by the \texttt{Goal}
argument. This can be a named binding, \texttt{goal :: Name -> Goal},
or the default goal \texttt{defaultGoal :: Goal}. The default goal is
the first goal generated by the most recent tactic application.

\subsubsection{Hole Manipulations}
There are three basic operations on holes, \demph{claim}, \demph{fill},
and \demph{abandon}; these are given the following types:

\begin{verbatim}
claim :: IsTerm a => Name -> a -> Tactic
fill :: IsTerm a => a -> Tactic
abandon :: Tactic
\end{verbatim}

The \hdecl{claim} function takes a name and a type and creates a new
hole. The \hdecl{fill} function takes a guess to attach to the current
goal. In addition, \hdecl{fill} attempts to solve other goals by
unification. Attaching a guess does not necessarily solve the goal
completely; if the guess contains further hole bindings, it cannot yet
have any computational force. 
%% The \hdecl{solve} tactic is provided to
%% check whether a guess is \demph{pure} (i.e. does not contain any hole
%% bindings or guesses itself) and converts it to a $\RW{let}$ binding if
%% so.  
A guess can be removed from a goal with the \hdecl{abandon}
tactic.

%% It can be inconvenient to have to \texttt{solve} every goal after a
%% \texttt{fill} (although sometimes this level of control is
%% useful). For this reason, \texttt{fill} and other tactics will
%% automatically solve all goals with hole-free guesses attached. More
%% fine-grained tactics are available, but are beyond the scope of this paper.

\subsubsection{Introductions}
A basic operation on terms is to introduce $\lambda$ bindings into the
context. The \texttt{intro} and \texttt{introName} tactics operate on
a goal of the form $\fbind{\vx}{\vS}{\vT}$, introducing
$\lam{\vx}{\vS}$ into the context and updating the goal to
$\vT$. \texttt{introName} allows a user specified name choice,
otherwise \Ivor{} chooses the name.

\begin{verbatim}
intro :: Tactic
introName :: Name -> Tactic
\end{verbatim}

For example, to define a unary addition function, we might begin
with

\DM{
\Axiom{
\hole{\VV{plus}}{\Nat\to\Nat\to\Nat}
}
}

Applying \texttt{introName} twice with the names $\vx$ and $\vy$ gives
the following proof state, with $\vx$ and $\vy$ introduced into the
local context:
\DM{
\Rule{
\AR{
\lam{\vx}{\Nat}\\
\lam{\vy}{\Nat}
}
}
{\hole{\VV{plus\_H}}{\Nat}}
}

\subsubsection{Refinement}
The \texttt{refine} tactic solves a goal by an application of a
function to arguments. Refining attempts to solve a goal of type
$\vT$, when given a term $\vt\Hab\fbind{\tx}{\tS}{\vT}$. The tactic
creates a subgoal for each argument $\vx_i$, attempting to solve it by
unification.

\begin{verbatim}
refine :: IsTerm a => a -> Tactic
\end{verbatim}

For example, given a goal
\DM{
\Axiom{
\hole{\vv}{\Vect\:\Nat\:(\suc\:\vn)}}
}

Refining by $\Vcons$ creates subgoals for all four arguments, and
attaches a guess to $\vv$:
\DM{
\Axiom{
\AR{
\hole{\vA}{\Type}\\
\hole{\vk}{\Nat}\\
\hole{\vx}{\vA}\\
\hole{\vxs}{\Vect\:\vA\:\vk}\\
\guess{\vv}{\Vect\:\Nat\:(\suc\:\vn)}{\Vcons\:\vA\:\vk\:\vx\:\vxs}
}
}
}

However, for $\Vcons\:\vA\:\vk\:\vx\:\vxs$ to have type
$\Vect\:\Nat\:(\suc\:\vn)$ requires that $\vA=\Nat$ and $\vk=\vn$.
Refinement unifies these, leaving the
following goals:
\DM{
\Axiom{
\AR{
\hole{\vx}{\Nat}\\
\hole{\vxs}{\Vect\:\Nat\:\vn}\\
\guess{\vv}{\Vect\:\Nat\:(\suc\:\vn)}{\Vcons\:\Nat\:\vn\:\vx\:\vxs}
}
}
}


\subsubsection{Elimination}
Refinement solves goals by constructing new values; we may also solve
goals by deconstructing values in the context, using an elimination
operator as described in section \ref{elimops}. The \texttt{induction}
and \texttt{case} tactics apply the $\delim$ and $\dcase$ operators
respectively to the given target:

\begin{verbatim}
induction, cases :: IsTerm a => a -> Tactic
\end{verbatim}

These tactics proceed by refinement by the appropriate elimination
operator. The motive is calculated automatically, and is given by the
goal to be solved. Each tactic generates subgoals for each method of
the appropriate elimination rule.

%% A more general elimination tactic is \texttt{by}, which takes an
%% application of an elimination operator to a target.

%% \begin{verbatim}
%% by :: IsTerm a => a -> Tactic
%% \end{verbatim}

%% The type of the term given to \texttt{by} must be a function expecting
%% a motive and methods.

An example of \texttt{induction} is in continuing the definition of
our addition function. This can be defined by induction over the first
argument. We have the proof state

\DM{
\Rule{
\AR{
\lam{\vx}{\Nat}\\
\lam{\vy}{\Nat}
}
}
{\hole{\VV{plus\_H}}{\Nat}}
}

Applying \texttt{induction} to $\vx$ leaves two subgoals, one for the
case where $\vx$ is zero, and one for the inductive case:

\DM{
\Rule{
\AR{
\lam{\vx}{\Nat}\\
\lam{\vy}{\Nat}
}
}
{
\AR{
\hole{\VV{plus\_O}}{\Nat}\\
\hole{\VV{plus\_S}}{\fbind{\vk}{\Nat}{\fbind{\VV{ih}}{\Nat}{\Nat}}}
}
}
}

By default, the next goal to solve is $\VV{plus\_O}$. However, the
\hdecl{focus} tactic can be used to change the default goal.

\subsubsection{Rewriting}
It is often desirable to rewrite a goal given an equality proof, to
perform equational reasoning. The \texttt{replace} tactic replaces
occurrences of the left hand side of an equality with the right hand
side. To do this, it requires four arguments:

\begin{enumerate}
\item The equality type; for example
  $\TC{Eq}\Hab\fbind{\vA}{\Type}{\vA\to\vA\to\Type}$.
\item A replacement lemma, which explains how to substitute one term
  for another; for example\\
  $\FN{repl}\Hab\fbind{\vA}{\Type}{
    \fbind{\va,\vb}{\vA}{
	\TC{Eq}\:\_\:\va\:\vb\to\fbind{\vP}{\vA\to\Type}{
	  \vP\:\va\to\vP\:\vb}}}$
\item A symmetry lemma, proving that equality is symmetric; for
  example\\
  $\FN{sym}\Hab\fbind{\vA}{\Type}{
      \fbind{\va,\vb}{\vA}{\TC{Eq}\:\_\:\va\:\vb\to\TC{Eq}\:\_\:\vb\:\va}}$
\item An equality proof.
\end{enumerate}

The \Ivor{} distribution contains a library of $\source$ code with the
appropriate definitions and lemmas. Requiring the lemmas to be
supplied as arguments makes the library more flexible --- for example,
heterogeneous equality~\cite{mcbride-thesis} may be preferred.  The
type of the \texttt{replace} tactic is:

\begin{verbatim}
replace :: (IsTerm a, IsTerm b, IsTerm c, IsTerm d) =>
               a -> b -> c -> d -> Bool -> Tactic
\end{verbatim}

The \texttt{Bool} argument determines whether to apply the symmetry
lemma to the equality proof first, which allows rewriting the right
hand side to the left hand side. This \hdecl{replace} tactic is
similar to \Lego{}'s \texttt{Qrepl} tactic \cite{lego-manual}.

For example, consider the following fragment of proof state:

\DM{
\Rule{
\AR{
\ldots\\
\lam{\vx}{\Vect\:\vA\:(\FN{plus}\:\vx\:\vy)}
}
}
{
\hole{\VV{vect\_H}}{\Vect\:\vA\:(\FN{plus}\:\vy\:\vx)}
}
}

Since $\FN{plus}$ is commutative, $\vx$ ought to be a vector of the
correct length. However, the type of $\vx$ is not convertible to the
type of $\VV{vect\_H}$. Given a lemma $\FN{plus\_commutes}\Hab
\fbind{\vn,\vm}{\Nat}{\TC{Eq}\:\_\:(\FN{plus}\:\vn\:\vm)\:(\FN{plus}\:\vm\:\vn)}$,
we can use the \texttt{replace} tactic to rewrite the goal to the
correct form. Applying \texttt{replace} to $\TC{Eq}$, $\FN{repl}$,
$\FN{sym}$ and $\FN{plus\_commutes}\:\vy\:\vx$ yields the following
proof state:

\DM{
\Rule{
\AR{
\ldots\\
\lam{\vx}{\Vect\:\vA\:(\FN{plus}\:\vx\:\vy)}
}
}
{
\hole{\VV{vect\_H}}{\Vect\:\vA\:(\FN{plus}\:\vx\:\vy)}
}
}

This is easy to solve using the \texttt{fill} tactic with $\vx$.

\subsection{Tactic Combinators}

\label{combinators}

\Ivor{} can be seen as an embedded domain specific language for
building tactics. The library provides a number of
combinators for building more complex tactics from the basic tactics
previously described. By providing an API for basic tactics and a
collection of combinators, it becomes easy to extend the library with
more complex domain specific tactics. We will see examples in
sections \ref{example1} and \ref{example2}.

\subsubsection{Sequencing Tactics}
There are three basic operators for combining two tactics to create a third:

\begin{verbatim}
(>->), (>+>), (>=>) :: Tactic -> Tactic -> Tactic
\end{verbatim}

\begin{enumerate}
\item The \hdecl{>->} operator constructs a new tactic by sequencing two
tactic applications to the \remph{same} goal.

\item The \hdecl{>+>} operator constructs a new tactic by applying the
  first, then applying the second to the next \remph{default} goal.

\item The \hdecl{>=>} operator constructs a new tactic by applying the first
tactic, then applying the second to every subgoal generated by the first.

\end{enumerate}

Finally, the \hdecl{tacs} function takes a list of tactics and applies
them in sequence to the default goal:

\begin{verbatim}
tacs :: Monad m => [Goal -> Context -> m Context] ->
                   Goal -> Context -> m Context
\end{verbatim}

Note that the type of this can be understood as \hdecl{[Tactic] ->
  Tactic}, but the Haskell typechecker requires that the same monad be
abstracted over all tactics.

\subsubsection{Handling Failure}
Tactics may fail (for example a refinement may be ill-typed). 
Recovering gracefully from a failure may be needed, for
example to try a number of possible ways of rewriting a term.
The \hdecl{try} combinator is an exception handling combinator which
tries a tactic, and chooses a second tactic to apply to the same goal
if the first tactic succeeds, or an alternative tactic if the first
tactic fails.
The identity tactic, \hdecl{idTac}, is often appropriate on success.

\begin{verbatim}
try :: Tactic -> -- apply this tactic
       Tactic -> -- apply if the tactic succeeds
       Tactic -> -- apply if the tactic fails
       Tactic
\end{verbatim}

%% \subsection{The Shell}

%% The \texttt{Ivor.Shell} module provides a command driven interface to
%% the library, which can be used for experimental purposes or for
%% developing a library of core lemmas for a domain specific task. It is
%% written entirely with the \texttt{Ivor.TT} interface but provides a
%% textual interface to the tactics. This gives, among other things, a
%% convenient method for loading proof scripts or libraries, or allowing
%% user directed proofs in the style of other proof assistants such as
%% \Coq{}.

%% A small driver program is provided (\texttt{jones}), which gives a
%% simple interface to the \Ivor{} shell.


\section{A First Order Logic Theorem Prover}

\label{example1}

\section{\Funl{}, a Functional Language with a Built-in Theorem Prover}

\label{example2}




\section{Related Work}

The ability to extend a theorem prover with user defined tactics has
its roots in Robin Milner's LCF~\cite{lcf-milner}. This introduced the
programming language ML to allow users to write tactics; we follow the
LCF approach in exposing the tactic engine as an API. However, unlike
other systems, we have not treated the theorem prover as an end in
itself, but intend to expose the technology to any Haskell application
which may need it.  The implementation of \Ivor{} is based on the
presentation of \Oleg{} in Conor McBride's
thesis~\cite{mcbride-thesis}; this technology also forms the basis for
the implementation of \Epigram{}~\cite{view-left}. The core language
of \Epigram{}~\cite{epireloaded} is similar to $\source$, with
extensions for observational equality. We use implementation
techniques from \cite{not-a-number} for dealing with variables and
renaming.

Other theorem provers such as \Coq{}~\cite{coq-manual},
\Agda{}~\cite{agda} and Isabelle~\cite{isabelle} have varying degrees
of extensibility. The interface design largely follows that of
\Coq{}. \Coq{} includes a high level domain specific language for
combining tactics and creating new tactics, along the lines of the
tactic combinators presented in section \ref{combinators}. This
language is ideal for many purposes, such as our \hdecl{contradiction}
tactic, but more complex examples such as \hdecl{buildTerm} would
require extending \Coq{} itself.  Isabelle~\cite{isabelle} is a
generic theorem prover, in that it includes a large body of object
logics and a meta-language for defining new logics. It includes a
typed, extensible tactic language, and can be called from ML programs,
but unlike \Ivor{} is not based on a dependent type theory.

The implementation of \Funl{} allows a theorem prover to be attached
to the language in a straightforward way, using \Ivor{}'s tactics
directly. This would be a possible method of attaching a theorem
prover to a more full featured programming language such as the
Sparkle~\cite{sparkle} prover for Clean~\cite{clean}.

\section{Conclusions}

We have seen an overview of the \Ivor{} library, including basic
tactics for building proofs similar to the tactics available in other
proof assistants such as \Coq{}. By exposing the tactic API and
providing an interface for term construction and evaluation, we are
able to embed theorem proving technology in a Haskell
application. This in itself is not a new idea, having first been seen
as far back as the LCF~\cite{lcf-milner} prover --- however, the
theorem proving technology is not an end in itself, but a
mechanism for constructing domain specific tools such as the
propositional logic theorem prover in section \ref{example1} and the
programming language with built in equational reasoning support in
section \ref{example2}.

The library includes several features we have not been able to discuss
here. There is experimental support for multi-stage programming with
dependent types, exploited in~\cite{dtpmsp-gpce}.  The term language
can be extended with primitive types and operations, e.g. integers and
strings with associated arithmetic and string manipulation
operators. Such features would be essential in a representation of a
real programming language. In this paper, we have stated that
$\source$ is strongly normalising, with no general recursion allowed,
but again in the representation of a real programming language general
recursion may be desirable --- however, this means that correctness
proofs can no longer be total. The library can optionally allow
general recursive definitions, but such definitions cannot be
evaluated by the typechecker. Finally, a command driven interface is
available, which can be accessed as a Haskell API or used from a
command line driver program, and allows user directed proofs in the
style of other proof assistants. These and other features are fully
documented on the web site.

\subsection{Further Work}

Development of the library has been driven by the requirements of
our research into Hume~\cite{Hume-GPCE}, a resource bounded functional
language. We are investigating the use of dependent types in
representing and verifying resource bounded functional
programs~\cite{dt-framework}. For this, automatic generation of
injectivity and disjointness lemmas for constructors will be
essential~\cite{concon}, as well as an elimination with a
motive~\cite{elim-motive} tactic. Future versions will include
optimisations from \cite{brady-thesis} and some support for compiling
$\source$ terms; this would not only improve the efficiency of the
library, but also facilitate the use of \Ivor{} in a real language
implementation. Finally, an implementation of coinductive
types~\cite{coinductive} is likely to be very useful; currently it can
be achieved by implementing recursive functions which do not reduce at
the type level, but a complete implementation with criteria for
checking productivity would be valuable for modelling streams in Hume.


\section*{Acknowledgements}

%This work is generously supported by EPSRC grant
%EP/C001346/1.

\bibliographystyle{abbrv}
\begin{small}
\bibliography{../bib/literature.bib}

%\appendix
%\input{coqscript}

\end{small}
\end{document}
